\documentclass[10pt, a4paper]{article}

\usepackage{multicol} % деление на колонки
\usepackage{fancyhdr} % настройка колонтитулов
\usepackage{parskip} % настройка отступов абзаца
\usepackage{newtxtext, newtxmath} % Times New Roman
\newcommand*{\defeq}{\stackrel{\text{def}}{=}}
\usepackage{indentfirst} % отступ после заголовка секции
\usepackage[hidelinks]{hyperref} % ссылки без контура
\usepackage{mathtools} % математические формулы
\usepackage{titlesec} % настройка заголовков
\usepackage[left=2.5cm,right=2.5cm, top=2.1cm,bottom=2.5cm]{geometry} % макет страницы
\fancyhf{} % очистка колонтитулов
\cfoot{\textbf{\thepage}} % жирные номера страниц
\pagestyle{fancy}
\renewcommand{\headrulewidth}{0pt} % удаление линии header
\setlength{\columnsep}{0.6cm} % расстояние между столбцами
\setlength{\parskip}{0pt} % вертикальный отступ абзаца
\setlength{\parindent}{0.5cm} % горизонтальный отступ абзаца
\setcounter{page}{29} % нумерация страниц
\titleformat{\section}{\normalsize\centering}{\thesection. }{0cm}{}[] % стиль заголовков разделов
\titlespacing*{\section} % отступ возле секций
{0pt}{0.6cm}{0.3cm}
\renewcommand{\thesection}{\Roman{section}} % римские цифры
\setcounter{section}{0} % нумерация секций
\begin{document}

\begin{center}
\huge{\textbf{Towards the Theory of Semantic Space}}\\
\vspace{0.4cm}
\normalsize
Valerian Ivashenko\\
\textit{Department of Intelligent Information Technologies\\
Belarusion State University of Informatics and Radioelectronics}\\
Minsk, Republic of Belarus\\
\href{mailto:ivashenko@bsuir.by}{whitivashenko@bsuir.by}
\end{center}
\begin{multicols}{2}
\textbf{\textit{Abstract}—The paper considers models for investigating
the structure, topology and metric features of a semantic
space using unified knowledge representation.}

\textbf{The classes of finite structures corresponding to ontological structures and sets of classical and non-classical kinds
are considered, and the enumerability properties of these
classes are investigated.}

\textbf{The notion of operational-information space as a model
for investigating the interrelation of operational semantics
of ontological structures of large and small step is proposed.}

\textbf{Quantitative features and invariants of ontological structures oriented to the solution of knowledge management
problems are considered.}

\textbf{\textit{Keywords}—Semantic Space, Neg-entropy, Operationalinformation space, Enumerable sets, Natural numbers,
Ackermann coding, Generalized formal language, Enumerable self-founded Hereditarily finite sets, Countable nonidentically-equal Hereditarily finite sets, Multigraph, Hypergraph, Metagraph, Orgraph, Unoriented graph, Quasimetric, Orgraph invariant, Dynamic graph system, Resonator, Graph dimension, Fully-connected orgraph period,
Rado graph, Universal model, Stable structure, Operational
semantics, Denotational Semantics, Infinite structures, Generalized Kleene closure}
\section{Introduction}

There are different approaches to the study of topological, metrical and other properties of signs in texts
leading to the consideration of corresponding semantic
(or meaning (sense)) spaces [56].

Space is convenient because it is connected with some
ordinal or metric scale which allows to reduce the cost of
solving such cognitive tasks as searching (synthesis) or
checking (analysis) the presence of an element (including
for the purpose of eliminating redundancy) in a set
organized as a space.

Knowledge integration based on unification is necessary both to eliminate redundancy and to compute semantic metrics. For this purpose, the developed model of
unified knowledge representation [1], [5] can be adopted.

\section{Approaches to the construction of a meaning space}

The history of the development of the concept of
“meaning space” and the corresponding models are described in the works [2], [11], [32], [56].

As stated in [56], the main approaches to the construction and research of the organization of meaning space
include:

\begin{itemize}
    \item exterior studying the physical nature [30], [33], [48]
of processes including thinking processes [29],
    \item (quantitative) interior using quantitative and soft
models, including probabilistic description of processes [11], [34], [35], [42], based on the practice of using words of language [20], [53], [48], [54]
    \item (qualitative) interior investigating the structure of
represented knowledge and its dynamics [12], using
formal semiotic models [51].
\end{itemize}
\noindent In some cases, it is possible to combine these elements
of these approaches.

The following models and methods are used to construct and investigate the semantic space:
\begin{itemize}
    \item mathematical models of spaces [37]–[41], [43],
    \item formal and generalized formal languages [45], [56],
    \item methods of probability theory [11], [36], [54], [55], [57]
    \item methods of formal concepts analysis [58], [59], [61]
    \item other models [3], [4], [45], [46], [49], [51], [54], [56].
\end{itemize}

Further in the paper we consider the main classes of
structures, their attributes and corresponding types of
subspaces of the semantic space using unified knowledge
representation [5], [12].
\section{ Unified representation and classification of fully
representable finite knowledge structures}

At the level of syntax, using syntactic links, it is
possible to represent only finite knowledge structures in
a unified (explicit) way.

Let us consider the principles of unified representation
of knowledge [5], [12] with a structure that is one of finite
structures of different kinds. Let us compare a certain
class of structures to each kind of finite knowledge
structures.

Note that finite structures can be divided into two main
types: oriented finite structures and unoriented finite
structures [21].

The simplest unoriented finite structures are hereditarily finite sets [63]. The structures are hereditarily class of hereditarily finite sets
can be expressed as follows:

\vspace{0.2cm}
\begin{center}
$\emptyset^{(+^{*}_{1})}=H_{\aleph_{0}}$
\end{center}
\end{multicols}
\begin{multicols}{2}
\noindent where
\begin{center}
\[\binom{0}{A}\defeq2^{\emptyset} = \{\emptyset\}\]\\
\[\binom{1}{A} \defeq\bigcup\limits_{x\in{A}} 2^{\{x\}}/\binom{0}{A}\]\\
\[\binom{\iota+1}{A}\defeq \left( \binom{\iota}{A}\textstyle{\bigcup\limits^\smile} \displaystyle{\binom{1}{A}} \right) / \binom{\iota}{A} \]
\[A\textstyle{\bigcup\limits^\smile}B\defeq \displaystyle\bigcup\limits_{\left\langle P,Q\right\rangle\in{A\times B}} \{P\cup U\}\]
\[2^{\left(\emptyset+\Sigma_{x \in A} \{x\} \right)} \defeq \bigcup\limits_{\iota \in \mathbb{N} \cup \{0\}}\binom{\iota}{A}\]
\[A^{(+^1_k)}\defeq \tau_k(\rho_k(\langle A, A\rangle)\cup \sigma_k (\langle A, A\rangle))\]
\[A^{(+^{\iota+1}_k)}\defeq \tau_k\Big(\Big(\rho_k\Big(\Big\langle A, A^{(+^\iota_k)}\Big\rangle\Big)\cup \sigma_k \Big(\Big\langle A, A^{(+^\iota_k)}\Big\rangle\Big)\]
\[A^{(+^*_k)}\defeq \bigcup\limits_{\iota\in\mathbb{N}/\{0\}}A^{(+^\iota_k)}\]
\[\tau_1(A)\defeq A\]
\[\rho_1(\langle A,B \rangle)\defeq \emptyset\]
\[\sigma_1(\langle A, B \rangle) \defeq 2^{\emptyset + \Sigma_{x\in(A\cup B)}\{x\}}\]
\end{center}

According to Ackermann coding [62], all hereditarily
finite sets can be a mutually uniquely matched to natural
numbers and thus enumerated [27]:
\begin{center}
\[f(S) = 0 + \sum\limits_{x\in S}2^{f(x)}\]
\end{center}
\noindent A generalization of the class of hereditarily finite sets is
the class of generalized hereditarily finite sets.
\begin{center}
\[A^{(+^*_1)}\]
\end{center}

Generalized hereditarily finite sets can be embedded
in (classical non-generalized) hereditarily finite sets:
\begin{center}
\[\emptyset\sim 2^\emptyset\]
\[a_k\sim 2^{{\{\emptyset\}}_k}\]
\[g(\emptyset) = \{\emptyset\}\]
\[g(a_k) = \{\{\emptyset\}_k, \emptyset\}\]
\[g(X) = \{g(x)\mid x\in X\}\]
\end{center}
\noindent or alternatively:
\begin{center}
\[\emptyset \sim d(1) = \{\emptyset\}\] 
\[a_k \sim d(2*k+1)\]
\[d(k) = \bigcup\limits_{i=1}^{\lfloor \log_2 k \rfloor}\Big\{d\Big(\Big\lfloor \frac{k}{2^i} \Big\rfloor \Big)mod2\Big\}\]
\[d(0) = \emptyset\]
\[d(1) = \{\emptyset\}\]
\[d(2) = \{\{\emptyset\}\}\]
\[d(3) = \{\{\emptyset\}, \emptyset\}\]
\[d(4) = \{\{\{\emptyset\}\}\}\]
\[d(5) = \{\{\{\emptyset\}\},\emptyset\}\]
\[d(6) = \{\{\{\emptyset\}\},\{\emptyset\}\}\]
\[d(7) = \{\{\{\emptyset\}\},\{\emptyset\},\emptyset\}\]
\[d(8)=\{\{\{\emptyset\}, \emptyset\}\}\]
\[d(9)=\{\{\{\emptyset\}, \emptyset\},\emptyset\}\]
\[d(10)=\{\{\{\emptyset\}, \emptyset\},\{\emptyset\}\}\]
\[d(11)=\{\{\{\emptyset\}, \emptyset\},\{\emptyset\},\emptyset\}\]
\[d(12)=\{\{\{\emptyset\}, \emptyset\},\{\{\emptyset\}\}\}\]
\[d(13)=\{\{\{\emptyset\}, \emptyset\},\{\{\emptyset\}\},\emptyset\}\]
\[d(14)=\{\{\{\emptyset\}, \emptyset\},\{\{\emptyset\}\},\{\emptyset\}\}\]
\[d(15)=\{\{\{\emptyset\}, \emptyset\},\{\{\emptyset\}\},\{\emptyset\},\emptyset\}\]
\[d(16)=\{\{\{\{\emptyset\}\}\}\}\]
. . . 
\[g(\emptyset) = \{\emptyset\}\]
\[g(a_k)=d(2*k+1)\]
\[g(X) = \{g(x)\mid x\in X\}\]
\end{center}


In this way we obtain an ordering of generalized
hereditarily finite sets (as example) in accordance with
the Ackermann numbering and embedding in hereditarily
finite (unoriented) sets.

As for oriented structures (oriented, “ordered” sets),
if we take the von Neumann-Bernays-Gödel axiomatics
[63] as a basis then with some “traditional” approach
(representation of oriented pairs according to K. Kuratowski) [24] an empty string [10], [13], [14], an empty
an oriented set [22] cannot be represented as unfounded
sets in a theory with the von Neumann-Bernays-Gödel
axiomatics [26], [63].

Accepted:

\begin{center}
\[x=\langle x \rangle\]
\end{center}
in this case, the oriented pair of K. Kuratovsky:
\begin{center}
\[\langle x, y \rangle = \{\{x\}, \{x,y\}\}\]
\end{center}
\noindent also
\begin{center}
\[\langle x_1,x_2,x_3 \rangle\ = \langle \langle x_1,x_2 \rangle ,x_3 \rangle\]
\[\langle x_1,x_2,x_3,x_4\rangle\ = \langle \langle x_1,x_2,x_3 \rangle ,x_4 \rangle\]
\[\langle x_1,x_2,x_3,x_4,x_5\rangle\ = \langle \langle x_1,x_2,x_3,x_4 \rangle ,x_5 \rangle\]
\end{center}
\end{multicols}
\begin{multicols}{2}
$\langle x_1,x_2,x_3,x_4,x_5,x_6\rangle\ = \langle \langle x_1,x_2,x_3,x_4,x_5\rangle ,x_6 \rangle$
\[\langle x_1,x_2, ..., x_i, ..., x_{n-1}, x_n \rangle =\langle\langle x_1, x_2, ..., x_i, ..., x_{n-1}\rangle,x_n\rangle\]
\[A^n = \{\langle x_1,x_2, ..., x_i, ..., x_{n-1}, x_n \rangle \mid x_i \in A\}\]
The consequence of this is that strings are conditionally
dimensional, that is, the length of a string is not its
function, and therefore cannot be calculated uniquely
from a string; an empty string cannot be represented by
a set in the von Neumann-Bernays-Gödel theory:
\begin{center}
\vspace{0.25cm}
$\langle x,x \rangle = \{\{x\}\} = \langle \{x\} \rangle$
\[2 = length(\langle x,x \rangle) \neq length(a) = 1\]
\[n = length(\langle x_1, x_2, ..., x_i, ..., x_{n-1}, x_n \rangle) \neq \] \[ length(\langle x_1, x_2, ..., x_i, ..., x_{n-1}, x_n \rangle) = 2\]
\end{center}
\vspace{0.25cm}
During understanding the string length function, if we
move from a function (as in the formulas above) to a
higher-order function with respect to the set of elements
of an oriented set this does not solve the problem:
\vspace{0.25cm}
\begin{center}
$2 = length(\{\langle x,x \rangle , \{x\}\})(\langle x,x \rangle) \neq $ \[length(\{\langle x,x \rangle , \{x\}\})(\langle \{x\} \rangle) = 1\]
\end{center}
Another consequence of this is that the Cartesian power
can exhibit the following non-obvious and non-intuitive
properties:
\begin{center}
    \[\exists A(A = A^1 \supset A^2 \supset A^3 \supset ... \supset A^i \supset ...)\]
\end{center}
Inability to represent the empty string $\varepsilon$ when representing strings as oriented sets

Let be:
\vspace{0.3cm}
\begin{center}
$E = \{\varepsilon\}$
\end{center}

Required:
\vspace{0.3cm}
\begin{center}
$E^n = E$
\end{center}

We have:
\begin{center}
\vspace{0.3cm}
$E^2 = \{\langle \varepsilon, \varepsilon \rangle\} = \{\{\{\varepsilon\}\}\}$
\[\varepsilon = \{\{\varepsilon\}\}\]
\[E^3 = \{\langle \varepsilon, \varepsilon, \varepsilon\rangle\} = \{\langle\langle \varepsilon, \varepsilon \rangle\}, \{\langle \varepsilon, \varepsilon \rangle, \varepsilon\}\}\}\]
\[E^3 = \{\{\{\{\{\varepsilon\}\}\}, \{\{\{\varepsilon\}\}, \varepsilon\}\}\}\]
$\varepsilon = \{\{\{\{\varepsilon\}\}\}, \{\{\{\varepsilon\}\}, \varepsilon\}\}$
\vspace{0.35cm}
\end{center}
The latter violates the axiom of regularity (foundation),
otherwise:
\begin{center}
$E^2 \neq E$
\[E^3 \neq E\]
\end{center}
The use of non-founded sets is evidence of a transition
to non-classical mathematical models


There are approaches to representing strings in
von Neumann-Bernays-Gödel set theory by equivalence
classes of groupoids (which is complex) over oriented
sets or functions (requires the construction of a set of ordinal numbers). In the first case, the representation grows
exponentially, and in the second case, it is necessary to
use oriented pairs [23], [25] (the number of characters for
a string of length n grows no faster than $1+14*n+p (n)$,
where $p(n) = 1 + n*(n + 3)/2$ – number of characters
to represent ordinal numbers). These approaches do not
require a transition to non-classical mathematical models.
However, a string of one element is not this element


Let’s consider another approach to representing strings
and oriented sets, which does not require, overcomes the
identified difficulties within the framework of classical
mathematical models and uses pairs not according to
K. Kuratowski, which cannot counter-intuitively have
cardinality (length) equal to on

Let us define the concept of disposing of a set
\begin{center}
\[1^S \defeq S\]
\[(\iota + 1)^S \defeq \{\iota^T \mid T \subseteq  S\}\]
\end{center}
Example.
\vspace{0.3cm}
\begin{center}
$2^{\{\chi\}} = \Big\{1^{\{\chi\}}, 1^\emptyset \Big\} = \{\{\chi\},\emptyset\}$
\[3^{\{\chi\}} = \Big\{2^{\{\chi\}}, 2^\emptyset \Big\} = \Big\{\{\chi\},\emptyset, \Big\{1^\emptyset \Big\}\Big\} = \{\{\{\chi\},\emptyset\}, \{\emptyset\}\]
\[4^{\{\chi\}} = \Big\{3^{\{\chi\}}, 3^\emptyset \Big\} = \{\{\{\{\{\chi\},\emptyset\},\{\emptyset\}\},\{\{\emptyset\}\}\}\]
\[5^{\{\chi\}} = \Big\{4^{\{\chi\}}, 4^\emptyset \Big\} = \{\{\{\{\{\{\chi\},\emptyset\},\{\emptyset\}\},\{\{\emptyset\}\}\}, \{\{\{\emptyset\}\}\}\}\]
\[4^\emptyset = \Big\{3^\emptyset\Big\} = \Big\{\Big\{2^\emptyset\Big\}\Big\} = \Big\{\Big\{\Big\{1^\emptyset\Big\}\Big\}\Big\} = \{\{\{\{\emptyset\}\}\}\}\]
\end{center}
\vspace{0.4cm}
Also, let us define the concept of an individual set.
\begin{center}
\[\{x\}_1 \defeq \{x\}\]
\[\{x\}_{\iota + 1} \defeq \{\{x\}_\iota\}\]
\end{center}
note that:
\begin{center}
    $(\iota + 1)^\emptyset = \{\emptyset\}$
\end{center}
Finite oriented set:
\begin{center}
\[\bigcup\limits^k_{i=1}\Big\{(k-i+1)^{a_i}\Big\}_i\]
\end{center}
The number of characters to represent it is no more than
$1 + n * (5 * n + 1) /2 + q (n)$ where $q(n) = 2 * n + 1$ is
the number of characters per representation of individual
sets of the empty set

Examples:
\vspace{0.4cm}
\begin{center}
$\langle \rangle = \emptyset$ \\
$\langle \chi \rangle = \{\{\chi\}\}$
\end{center}
\end{multicols}
\end{document}
